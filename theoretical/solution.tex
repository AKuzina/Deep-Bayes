\documentclass[]{article}
%% Language and font encodings
\usepackage[english]{babel}
\usepackage[utf8x]{inputenc}
\usepackage[T1]{fontenc}

%% Sets page size and margins
\usepackage[a4paper,top=2cm,bottom=2.5cm,left=2cm,right=1.5cm,marginparwidth=1.75cm]{geometry}
\usepackage{setspace}
\setstretch{1.2}
%% Useful packages

\usepackage{amsthm}
\usepackage{amsfonts}
\usepackage{amsmath, amssymb}
\usepackage{graphicx}
\usepackage{subcaption}
\usepackage{bm}
\usepackage[colorinlistoftodos]{todonotes}
\usepackage[colorlinks=true, allcolors=blue]{hyperref}
\usepackage{float}


\title{Deep Bayes. Theoretical Tasks}
\author{Anna Kuzina}

\begin{document}

\maketitle



\section*{Problem 1}
R.v. $\xi \sim Pois(\lambda)$, if $\xi = k $ we perform k Bernoulli trials with the probability of success p. \\
R.v. $\eta$ ---  number of successful outcomes of Bernoulli trials.\\
Prove that $\eta \sim Pois(\lambda p)$.\\

\textbf{Proof}\\
\begin{align*}
\mathbb{P}(\eta = k) = \mathbb{P}(\xi = k)\cdot p^k + \mathbb{P}(\xi = k+1)\cdot p^{k}(1-p)\cdot (k+1) + \mathbb{P}(\xi = k+2)\cdot  p^{k}(1-p)^2\cdot C_{k+1}^2 + \dots =\\
=\sum_{i=0}^{\infty} \mathbb{P}(\xi = k+i) p^k (1-p)^{i}  C_{k+i}^i = 
\sum_{i=0}^{\infty}\left(   \frac{\lambda^{k+i}e^{-\lambda}}{(k+i)!}  p^k  (1-p)^{i}  \frac{(k+i)!}{i! k!}  \right)= 
\frac{e^{-\lambda}\lambda^k p^k}{k!}  \sum_{i=0}^{\infty}  \frac{\lambda^i (1-p)^i}{i!}
\end{align*}

The sum in the last term is exactly Taylor expansion of the exponent. Therefore:

\begin{align*}
\mathbb{P}(\eta = k) = \frac{e^{-\lambda}\lambda^k p^k}{k!}  e^{\lambda(1-p)} = 
\frac{(\lambda p)^k e^{-\lambda p}}{k!}
\end{align*}

Which means, that $\eta \sim Pois(\lambda p)$

\section*{Problem 2}
Strict reviewer: $t_1 \sim \mathcal{N}(30, 100)$\\
Kind reviewer:  $t_2 \sim \mathcal{N}(20, 25)$\\
Reviewer is chosen with prob 0.5.
Find $\mathbb{P}(kind | t = 10)$.\\

\textbf{Solution}
\begin{align*}
\mathbb{P} (kind | t = 10) = \frac{\mathbb{P} (kind \cap t = 10)}{\mathbb{P} (t = 10)} = \frac{0.5 \mathbb{P}(t_2 = 10) }{0.5 \mathbb{P} (t_2 = 10) + 0.5 \mathbb{P} (t_1 = 10)}
\end{align*}

Since, density of the normal distribution is known, we can easily calculate this probability:
\begin{align*}
\mathbb{P} (kind | t = 10) = \frac{0.5 \frac15 \exp(-\frac{(20 - 10)^2}{2\cdot 25})}{0.5 \frac15 \exp(-\frac{(20 - 10)^2}{2\cdot 25}) + 0.5 \frac{1}{10} \exp(-\frac{(30 - 10)^2}{2\cdot 100})} = \frac{0.1 \exp(-2)}{0.1 \exp(-2) + 0.05\exp(-2)} = \frac{10}{15} = \frac23
\end{align*}




\end{document}
